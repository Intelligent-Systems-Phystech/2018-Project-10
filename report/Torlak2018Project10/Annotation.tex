\documentclass{article}
\usepackage{blindtext}
\usepackage[utf8]{inputenc}
\usepackage[T2A]{fontenc}

\usepackage[left=2cm, right=3cm,
    top=2cm, bottom=2cm, bindingoffset=0cm]{geometry}
 \begin{document}

\title{\textbf{Сравнение нейросетевых и непрерывно-морфологических методов в задаче детекции текста (Text Detection)}}
\author{Гайдученко Н.Е., Труш Н.А, \textbf{Торлак А.В}, Миронова Л.Р., Акимов К.М., Гончар Д.А.}
\maketitle
\Large
 Данная статья посвящена описанию проекта по сравнению нейросетевых и непрерывно-морфологических методов в задаче считывания информации с изображений. В работе анализируются как алгоритмы , требующие большого числа выборок, так и методы, использующие небольшое количество данных. Так же решена задача определения границ данных методов.

\bigskip\textbf{Ключевые слова:} \textit{нейронные сети, непрерывно-морфологические методы, распознавание текста, распознавание изображений}
\end{document}